% Every Latex document starts with a documentclass command
\documentclass[a4paper, 11pt]{article}

% Load some packages
\usepackage{graphicx} % This allows you to put figures in
\usepackage{natbib}   % This allows for relatively pain-free reference lists
\usepackage[left=3cm,top=3cm,right=3cm]{geometry} % The way I like the margins
\usepackage{dsfont}
\usepackage{amsmath}


% This helps with figure placement
\renewcommand{\topfraction}{0.85}
\renewcommand{\textfraction}{0.1}

% Set values so you can have a title
\title{New Paper}
\author{Me}
\date{\today}



% Document starts here
\begin{document}

%% Actually put the title in
%\maketitle

%\abstract{This is the abstract}

% Need this after the abstract
\setlength{\parindent}{0pt}
\setlength{\parskip}{8pt}

\section{Bright Sky Example}
This problem actually gave me headaches during my PhD until
I thought harder about priors and realised why it was happening.
Consider an $N \times N$ image of a patch of sky. Without any noise,
we'd observe the `true' image, defined by the flux $f$ in each pixel:
\begin{align}
\textnormal{true image} &= \left\{f_{ij}\right\}
\end{align}
The true total flux of the patch of sky is
\begin{align}
F &= \sum_{i=1}^N \sum_{i=1}^N f_{ij}.
\end{align}

However, due to noise, we observe a perturbed version of the
flux in each pixel. i.e. our data is noisy fluxes $D_{ij}$,
such that
\begin{align}
p(D_{ij} | f_{ij}) &\sim \textnormal{Normal}\left(f_{ij}, \sigma^2\right)
\end{align}
Assuming $\sigma$ is known, the likelihood
we'd need to infer the $f$s from the $D$s
is therefore
\begin{align}
p\left(\{D_{ij}\} | \{f_{ij}\}\right)
&\propto \exp
\left[
  -\frac{1}{2\sigma^2}\sum_{i=1}^N\sum_{j=1}^N
                  \left(D_{ij} - f_{ij}\right)^2
\right].
\end{align}

With this likelihood, we can infer the $f$s
(the de-noised image) from the $D$s (the noisy data).
We know fluxes are non-negative, so $f_{ij} \geq 0$.

With an implicit uniform prior on the $f$s,
try MCMC sampling and using the chain of $f$s to
get a posterior for $F$. You should notice something
``wrong'' with it, caused by the uniform prior.



\bibliographystyle{plainnat}
\bibliography{references}{}


\end{document}

