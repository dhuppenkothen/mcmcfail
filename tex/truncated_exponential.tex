% Every Latex document starts with a documentclass command
\documentclass[a4paper, 11pt]{article}

% Load some packages
\usepackage{graphicx} % This allows you to put figures in
\usepackage{natbib}   % This allows for relatively pain-free reference lists
\usepackage[left=3cm,top=3cm,right=3cm]{geometry} % The way I like the margins
\usepackage{dsfont}
\usepackage{amsmath}


% This helps with figure placement
\renewcommand{\topfraction}{0.85}
\renewcommand{\textfraction}{0.1}

% Set values so you can have a title
\title{New Paper}
\author{Me}
\date{\today}



% Document starts here
\begin{document}

%% Actually put the title in
%\maketitle

%\abstract{This is the abstract}

% Need this after the abstract
\setlength{\parindent}{0pt}
\setlength{\parskip}{8pt}

\section{Truncated Exponential Example}
This example is inspired by Steve Gull, via David MacKay.
Suppose an astronomical transient
occurs at time $t=0$, and has a brightness that
decays exponentially over time:

\begin{align}
\textnormal{Brightness}(t) &=
\left\{
\begin{array}{lr}
Ae^{-\frac{t}{L}}, & t \geq 0\\
0,                 & \textnormal{otherwise}.
\end{array}
\right.
\end{align}

Suppose we observe the object between times $t=t_{\rm min}$ and
$t=t_{\rm max}$, and we want to know the value of $L$, but don't
particularly care about $A$. We observe $N$ photons, and their
arrival times $\{t_1, t_2, ..., t_N\}$.
The probability distribution for the arrival times
$\{t_i\}$
given $L$ (and $N$, which we consider to be prior information)
is
\begin{align}
p(t_1, t_2, ..., t_N | L)
&\propto \prod_{i=1}^N e^{-t_i/L}.
\end{align}
This is the probability distribution for the data given the
parameters, which will give us the likelihood function once
we plug in the observed data. However, we need to normalise
it first, as the observed photon arrival times must be between
$t_{\rm min}$ and $t_{\rm max}$:
\begin{align}
p(t_1, t_2, ..., t_N | L)
&= \prod_{i=1}^N
\frac{e^{-t_i/L}}
     {\int_{t_{\rm min}}^{t_{\rm max}} e^{-t_i/L} dt_i}\\
&= \prod_{i=1}^N
\frac{e^{-t_i/L}}
     {L\left[e^{-t_{\rm min}/L} - e^{-t_{\rm max}/L}\right]}\\
&= L^{-N}\left[e^{-t_{\rm min}/L} - e^{-t_{\rm max}/L}\right]^{-N}
e^{-\frac{\sum_{i=1}^N t_i}{L}}.
\end{align}






\bibliographystyle{plainnat}
\bibliography{references}{}


\end{document}

